\documentclass{article}

\usepackage[a4paper, total={6in, 8in}]{geometry}
\usepackage{amsmath}
\usepackage{siunitx}

\title{Termodinamica}
\author{Eugenio Barbieri Viale}

\begin{document}
\setlength{\parindent}{0pt}
\maketitle

\section*{Teoria cinetica}
$$ n = \frac{N}{N_a} \hspace{1em} n = \frac{m_{totale}}{m_{atomica}} $$
La massa totale è espressa in grammi, mentre quella atomica in dalton ($1u = \num{1.66e-27} kg$)

\vspace{1em}

Per un gas monoatomico:
\begin{align*}
    pV &= \frac{2}{3}N\overline{K} \\
    \overline{K} &= \frac{3}{2}kT \\
    v_{qm} &= \sqrt{\frac{3kT}{m}}
\end{align*}
Dove $m$ è la massa di una molecola, $T$ la temperatura e $k = \num{1.38e-23} J/K$

\vspace{1em}

In un gas a temperatura $T$, ogni grado di libertà di una molecola è associato a un'energia media pari a:
$$ \frac{1}{2}kT $$

\vspace{1em}

L'energia interna di un gas monoatomico (1) e di un gas biatomico (2):
\begin{align}
    U &= \frac{3}{2}nRT \hspace{1em} \text{3 gradi di libertà}\\
    U &= \frac{5}{2}nRT \hspace{1em} \text{5 gradi di libertà (vibrazione non considerata)}
\end{align}

\section*{Termodinamica}

\subsection*{Primo principio}
$$ \Delta U = Q - L $$
\begin{itemize}
    \item $Q > 0$: il sistema assorbe calore
    \item $Q < 0$: il sistema cede calore
    \item $L > 0$: il sistema compie lavoro 
    \item $Q < 0$: il lavoro è compiuto dall'ambiente sul sistema
\end{itemize}

\section*{Trasformazioni termodinamiche}
Trasformazione isobara (pressione costante):
$$ L = p\Delta V $$

\vspace{1em}

Trasformazione isocora (volume costante):
$$ L = 0 \longrightarrow \Delta U = Q $$

\vspace{1em}

Trasformazione isoterma (temperatura costante):
$$ L = nRT\ln{\frac{V_f}{V_i}} $$
$$ \Delta U = 0 \longrightarrow Q = L $$

\vspace{1em}

Trasformazione adiabatica (non viene scambiato calore):
$$ L = \frac{3}{2}nR(T_i - T_f) $$
$$ Q = 0 \longrightarrow \Delta U = -L $$

\section*{Calore specifico}
$$ Q = C_mn\Delta T $$
dove $C_m$ è il calore specifico molare ($J/(mol \cdot K)$)

\vspace{1em}

Gas monoatomico: 
\begin{align*}
    C_p &= \frac{5}{2}R \hspace{1em} \text{pressione costante} \\
    C_v &= \frac{3}{2}R \hspace{1em} \text{volume costante}
\end{align*}

\vspace{1em}

Gas biatomico:
\begin{align*}
    C_p &= \frac{7}{2}R \hspace{1em} \text{pressione costante} \\
    C_v &= \frac{5}{2}R \hspace{1em} \text{volume costante}
\end{align*}

Per un gas perfetto monoatomico:
$$ \gamma = \frac{5}{3} $$
Per un gas perfetto biatomico: 
$$ \gamma = \frac{7}{5} $$

\subsection*{Relazioni tra grandezze in trasformazioni adiabatiche}
\begin{align*}
    p_i(V_i)^{\gamma} &= p_f(V_f)^{\gamma} \\
    T_i(V_i)^{\gamma-1} &= T_f(V_f)^{\gamma-1} \\
    (p_i)^{1-\gamma}(T_i)^{\gamma} &= (p_f)^{1-\gamma}(T_f)^{\gamma}
\end{align*}

\section*{Macchine termiche}
$$ Q_c > 0 \hspace{1em} Q_f < 0 \hspace{1em} L > 0 $$
Rendimento di una macchina termica:
$$ \eta = \frac{L}{Q_c} $$
Dato che
$$ L = Q_c + Q_f = Q_c - |Q_f| $$
il rendimento può essere scritto anche:
$$ \eta = 1 - \frac{|Q_f|}{Q_c} $$

\vspace{1em}

La macchina di Carnot è una macchina termica reversibile. Il suo rendimento:
$$ \frac{|Q_f|}{Q_c} = \frac{T_f}{T_c} $$
$$ \eta_{carnot} = 1 - \frac{T_f}{T_c} $$
Questo è il rendimento massimo che una macchina termica può raggiungere operando tra le due temperature.

\vspace{1em}

Il ciclo di Carnot è composto da:
\begin{itemize}
    \item espansione isoterma a temperatura costante $T_c$
    \item espansione adiabatica: la temperatura diminuisce da $T_c$ a $T_f$
    \item compressione isoterma a temperatura costante $T_f$
    \item compressione adiabatica: la temperatura ritorna $T_c$
\end{itemize}

\section*{Macchine frigorifere e pompe di calore}
$$ Q_c < 0 \hspace{1em} Q_f > 0 \hspace{1em} L < 0 $$

Qui vale la relazione:
$$ |Q_c| = |L| + Q_f $$

Macchina frigorifera:
$$ COP = \frac{Q_f}{|L|} $$

Pompa a calore:
$$ COP = \frac{|Q_c|}{|L|} $$

\section*{Entropia}
In una trasformazione reversibile:
$$ \Delta S = \frac{Q}{T} $$
L'energia dell'universo non cambia in processi reversibili, mentre aumenta sempre in processi irreversibili. Questi ultimi causano un degrado dell'energia, una cui parte non è più utilizzabile. Il lavoro inutilizzato:
$$ L_{inut} = T_f \Delta S_{univ} $$

\end{document}

