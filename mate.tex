\documentclass[]{article}
\usepackage{amsmath, amssymb}
\usepackage{graphicx}

\title{Circonferenza, Ellisse e Iperbole}
\author{Eugenio Barbieri Viale}
\date{24 febbraio 2024}

\begin{document}
\maketitle

\section{Circonferenza}
\begin{itemize}
	\item La circonferenza è il luogo di punti equidistanti da un centro C.
	\item Forma centro-raggio con centro $C(x_0;y_0$) e raggio r:
		$$(x-x_0)^2+(y-y_0)^2=r^2$$
		Forma canonica:
		$$x^2+y^2+ax+by+c=0$$
	\item Per la forma canonica:
		$$C\left(-\frac{a}{2},-\frac{b}{2}\right) \hspace{3em} r=\sqrt{\frac{a^2}{4}+\frac{b^2}{4}-c}$$
	\item Condizione di realtà:
		$$\frac{a^2}{4}+\frac{b^2}{4}-c \ge 0$$
		I coefficienti di $x^2$ e $y^2$, se non sono uguali a 1, devono essere uguali tra loro ma diversi da zero.
		Se $r=0$ allora la circonferenza è degenere.
	\item Casi particolari:
		$$ a=0 \rightarrow \text{Il centro è sull'asse y}$$
		$$ b=0 \rightarrow \text{Il centro è sull'asse x}$$
		$$ c=0 \rightarrow \text{La circonferenza passa per l'origine}$$
		$$ a=0 \wedge c=0 \rightarrow r=C_y$$
		$$ b=0 \wedge c=0 \rightarrow r=C_x$$
	\item Posizioni reciproche tra una retta e una circonferenza (d è la distanza tra la retta e il centro):
		$$\text{Retta esterna:} \hspace{1em} d > r, \hspace{1em} \Delta < 0$$
		$$\text{Retta tangente:} \hspace{1em} d = r, \hspace{1em} \Delta = 0$$
		$$\text{Retta secante:} \hspace{1em} d < r, \hspace{1em} \Delta > 0$$
	\item Per trovare le tangenti passanti per un punto $P(x_0,y_0)\not\in\gamma$ a una circonferenza si può porre $\Delta=0$ oppure porre la distanza retta-centro uguale al raggio. La distanza centro-retta:
		$$d=\frac{|ax_c+by_c+c|}{\sqrt{a^2+b^2}}$$
	\item Per trovare le tangenti passanti per un punto $P(x_0,y_0)\in\gamma$ a una circonferenza la retta deve passare per P ed essere perpendicolare al raggio ($m_{raggio} m_{retta}=-1$) oppure si può usare la formula di sdoppiamento, sostituendo $x^2=xx_0$ e $y^2=yy_0$, mentre $x=\frac{x+x_0}{2}$ e  $y=\frac{y+y_0}{2}$. In questo modo:
		$$xx_0+yy_0+a\frac{x+x_0}{2}+b\frac{y+y_0}{2}+c=0$$
	\item Le posizioni reciproche tra due circonferenze sono 6: secanti, tangenti interne, tangenti esterne, interne, esterne, concentriche. Le due circonferenze sono concentriche quando hanno uguale a e b. Sottraendo membro a membro le due equazioni, si ottiene una retta detta asse radicale. L'asse radicale passa per i punti di intersezione se le circonferenze sono secanti, per il punto di tangenza se sono tangenti. Inoltre, esso è perpendicolare alla retta passante per i centri.
	\item Un fascio di circonferenze è la combinazione lineare tra due generatrici. Quella moltiplicata per k è la circonferenza esclusa. L'asse radicale è la circonferenza degenere del fascio. Se il fascio è di circonferenze tangenti, anche il punto base è una circonferenza degenere. L'asse radicale è perpendicolare all'asse centrale, ovvero la retta passante per i centri delle circonferenze.
\end{itemize}

\section{Ellisse}
\begin{itemize}
	\item L'ellisse è il luogo di punti P tali che sia costante la somma delle distanze di P dai due fuochi. In particolare, la somma è uguale a 2a. L'ellisse è la dilatazione di una circonferenza.
	\item L'equazione:
		$$\frac{x^2}{a^2}+\frac{y^2}{b^2}=1$$
		dove $a$ è la metà della distanza dei vertici su $x$, $b$ dei vertici su $y$. La distanza focale è $c$.
	\item Se i fuochi sono su $x$:
		$$ a > b \hspace{2em} c^2 = a^2-b^2 \hspace{2em} e=\frac{c}{a}=\frac{\sqrt{a^2-b^2}}{a}$$
		Se i fuochi sono su y:
		$$ a < b \hspace{2em} c^2 = b^2-a^2 \hspace{2em} e=\frac{c}{b}=\frac{\sqrt{b^2-a^2}}{b}$$
	\item L'eccentricità $e$ indica quanto è schiacciata l'ellisse. Maggiore è $e$, maggiore è lo schiacciamento sull'asse maggiore. Inoltre $0 \le e < 1$.
	\item Le posizione reciproche tra un'ellisse e una retta sono 3: secante, tangente, esterna.
	\item Per trovare la tangente all'ellisse in suo punto $P(x_0;y_0)\in\gamma$ bisogna usare la formula di sdoppiamento. Si ottiene sostituendo $x^2=xx_0$ e $y^2=yy_0$:
		$$\frac{xx_0}{a^2}+\frac{yy_0}{b^2}=1$$
	\item Per trovare le tangenti a un'ellisse passanti per un punto $Q(x_q;y_q)\not\in\gamma$ immaginiamo di far scorrere lungo l'ellisse un punto $P(x_0;y_0)$ insieme alla tangente alla stessa in quel punto. La tangente giusta sarà quella che passerà anche per $Q$. Per fare ciò bisogna risolvere il sistema:
		\begin{equation}
			\left\{
				\begin{aligned}
					&\frac{x_0^2}{a^2}+\frac{y_0^2}{b^2}=1\\
					&\frac{x_0x_q}{a^2}+\frac{y_0y_q}{b^2}=1
				\end{aligned}
			\right.
		\end{equation}
		dove le incognite sono $x_0$ e $y_0$. Una volta risolto si avranno i punti di tangenza. Per trovare le equazioni delle rette, basta imporre il passaggio per P e per Q.
	\item Un ellisse traslata di vettore $\vec{v}(x_c;y_c)$ ha centro $C(x_c;y_c)$ e ha equazione:
		$$\frac{(x-x_c)^2}{a^2}+\frac{(y-y_c)^2}{b^2}=1$$
		Ci si riconduce a questa forma utilizzando il completamento del quadrato.
	\item L'area racchiusa da un'ellisse è $A=\pi ab$

\end{itemize}
\end{document}

