\documentclass{article}

\usepackage[a4paper, total={6in, 8in}]{geometry}
\usepackage{amsmath}
\usepackage{textcomp}

\title{Onde}
\author{Eugenio Barbieri Viale}

\begin{document}
\setlength{\parindent}{0pt}
\maketitle

\section*{Introduzione}
\subsubsection*{Cos'è un'onda?}
Un'onda è una perturbazione che si propaga nello spazio e trasporta energia senza che ci sia un trasporto di materia

\subsubsection*{Diversi tipi di onda}
\begin{itemize}
    \item \textbf{onda trasversale}: le particelle del mezzo in cui si propaga l'onda oscillano perpendicolarmente alla direzione di propagazione
    \item \textbf{onda longitudinale}: in un solido elastico, le particelle del mezzo in cui si propaga l'onda oscillano lungo la direzione di propagazione \textit{(come il suono)}
\end{itemize}

\subsubsection*{Caratteristiche}
\begin{align*}
    \lambda &= \text{lunghezza d'onda} = \text{distanza tra due creste} \\
    T &= \text{periodo} = \text{$\Delta t$ in cui viene compiuta un'oscillazione completa} \\
    f &= \text{frequenza} = \frac{1}{T} \\
    v &= \text{velocità di propagazione} = \frac{\lambda}{T} = \lambda f \\
    v &= \sqrt{\frac{F_t}{\mu}} \hspace{1em} \text{con} \hspace{1em} \mu = \frac{m}{L} = \text{densità lineare}
\end{align*}

\subsubsection*{Descrizione matematica di un'onda}
$$ y = A \sin{\left(\omega t \pm kx\right)} $$
$$ \text{dove} \hspace{1em} \omega = \frac{2\pi}{T} \hspace{1em} \text{e} \hspace{1em} k = \frac{2\pi}{\lambda} $$
\begin{itemize}
    \item \textbf{$+$} onda si propaga verso sinistra (\textit{direzione} $-x$)
    \item \textbf{$-$} onda si propaga verso destra (\textit{direzione} $+x$)
\end{itemize}

\subsubsection*{Teorema di Fourier}
\textbf{enunciato}: \textit{Qualsiasi funzione periodica con frequenza $f$ può essere scritta come somma di funzioni sinusoidali con frequenze che sono multipli di $f$}

\section*{Il suono}
\subsubsection*{L'ampiezza massima}
$$ \Delta p_{max} = 2\pi fdvA $$
in cui $f$ è la frequenza, $d$ è la densità del mezzo, $v$ è la velocità di propagazione dell'onda, $A$ è lo spostamento massimo di una molecola dalla posizione di equilibrio

\subsubsection*{Velocità del suono in un gas}
$$ v_{suono} = \sqrt{\gamma k_b \frac{T}{m}} $$
$$ \text{dove} \hspace{1em} \gamma = \frac{c_p}{c_v} $$
\begin{itemize}
    \item $\gamma$ è il rapporto tra il calore specifico molare a pressione costante ($c_p$) e a volume costante ($c_v$)
    \item \textbf{gas monoatomico} \textrightarrow \hspace{1em} $\gamma = \frac{5}{3}$
    \item \textbf{gas biatomico} \textrightarrow \hspace{1em} $\gamma = \frac{7}{5}$
\end{itemize}

\subsubsection*{Intensità del suono}
$$ I = \frac{P}{A} = \left[\frac{W}{m^2}\right] $$
dove $P$ è la potenza sonora che attraversa perpendicolarmente una data superficie, $A$ è l'area della superficie

\vspace{1em}

$$ I = \frac{P}{4\pi r^2} $$
se la sorgente emette onde sonore in maniera isotropa vale questa relazione. La superficie $A$ è quella di una sfera e $r$ è il raggio di essa, ovvero la distanza dalla sorgente

\vspace{1em}

$$ I = \frac{\Delta p_{max}^2}{2dv} $$
$$ I = 2\pi^2 f^2 dvA^2 $$

\subsubsection*{Livello di intensità sonora}
$$ \beta = 10 \log\frac{I}{I_0} $$
con $I_0$ la soglia minima di intensità sonora udibile

\subsubsection*{Effetto Doppler}
\begin{align*}
    f_r &= f_s \frac{1}{1 - \frac{v_s}{v}} \rightarrow \text{sorgente si avvicina a ricevitore fermo ($f_r$ \textit{aumenta)}} \\
    f_r &= f_s \frac{1}{1 + \frac{v_s}{v}} \rightarrow \text{sorgente si allontana da ricevitore fermo($f_r$ \textit{diminuisce)}} \\
    f_r &= f_s(1 + \frac{v_r}{v}) \rightarrow \text{ricevitore si avvicina a sorgente ferma ($f_r$ \textit{aumenta)}} \\
    f_r &= f_s(1 - \frac{v_r}{v}) \rightarrow \text{ricevitore si allontana da sorgente ferma ($f_r$ \textit{diminuisce)}}
\end{align*}
Caso generale:
$$ f_r = f_s\left(\frac{1 \pm \frac{v_r}{v}}{1 \pm \frac{v_s}{v}}\right) $$

\subsubsection*{Diffrazione}
$$ \sin{\theta} = \frac{\lambda}{D} $$
dove $\theta$ è l'angolo di diffrazione, $D$ è la larghezza della fenditura attraverso la quale il suono passa

\section*{Onde stazionarie}
\textit{Le onde stazionarie sono onde che non si propagano ma rimangono confinate in una regione}

\subsubsection*{Trasversali}
$$ f_n = n \frac{v}{2L} \hspace{1em} n = 1,2,3,\dots $$
Queste frequenze costituiscono la serie armonica. Le onde hanno $n$ ventri

\subsubsection*{Longitudinali}
$$ f_n = n \frac{v}{2L} \hspace{1em} n = 1,2,3,\dots $$
il tubo ha lunghezza $L$ e ha le estremità aperte

\vspace{1em}
$$ f_n = (2n-1) \frac{v}{4L} \hspace{1em} n = 1,2,3,\dots $$
il tubo ha lunghezza $L$ e ha un'estremità aperta

\end{document}
