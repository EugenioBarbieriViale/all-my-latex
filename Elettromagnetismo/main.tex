\documentclass[12pt]{article}

\usepackage[a4paper, total={6in, 8in}]{geometry}
\usepackage{amsmath}

\title{Elettromagnetismo}
\author{Eugenio Barbieri Viale}
\date{5 gennaio 2026}

\begin{document}
\setlength{\parindent}{0pt}
\maketitle

\section*{Autoinduzione}
$$ \Phi(\vec{B}) = LI \hspace{1em} $$
Dove $\Phi(\vec{B})$ è il flusso e $L$ è l'induttanza, in henry (H)

$$ \text{fem} = -L \frac{dI}{dt} $$
Dove la fem è la forza elettromotrice

$$ L = \mu_0 \mu_r \frac{N^2}{l} A $$
Dove $L$ è l'induttanza di un solenoide e $\mu_r$ è la permeabilità relativa del materiale

$$ I(t) = \frac{V}{R}(1 - e^{-\frac{t}{L/R}}) $$
Dove $I$ è la corrente in un circuito $RL$ con tensione continua che si genera chiudendo il circuito. Inoltre la costante di tempo si definisce come $\tau = \frac{L}{R}$

$$ I(t) = \frac{V}{R}e^{-\frac{t}{L/R}} $$
Dove $I$ è la corrente autoindotta che si genera aprendo il circuito. Si chiama extracorrente di apertura

\section*{Mutua induzione}
$$ \text{fem}_s = -M \frac{dI_p}{dt} $$
Dove la fem è la forza elettromotrice che la spira primaria (con corrente variabile $I_p$) genera sulla spira secondaria

\section*{Energia immagazzinata in un campo magnetico}
$$ U = \frac{1}{2}LI^2 $$
Dove $U$ è l'energia immagazzinata in un induttore con induttanza $L$

$$ \rho = \frac{U}{\text{vol}} = \frac{1}{2\mu_0 \mu_r }B^2 $$
Dove $\rho$ è la densità di energia immagazzinata (per unità di volume)

\end{document}
