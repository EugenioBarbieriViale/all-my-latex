\documentclass[]{article}
\usepackage{hyperref}

\title{La Scorciatoia - Nello Cristianini}
\author{Eugenio Barbieri Viale}
\date{Estate 2024}

\begin{document}
\maketitle

\begin{itemize}
    \item \textbf{Perchè la conferenza di Dartmouth è stata così importante?} \par
        La conferenza di Dartmouth si tenne nel 1956 ed è considerata l'evento che ha segnato la nascita del campo dell' intelligenza artificiale. Qui venne presentato il primo programma per riprodurre artificialmente le capacità cognitive e di imparare proprie dell'uomo.

        \href{https://aitoolsexplorer.com/ai-history/the-dartmouth-conference-the-event-that-shaped-ai-research/}{\underline{The Dartmouth Conference: The Event that Shaped AI Research}}

    \item \textbf{Cosa sono le Lisp Machines?} \par
        Le \textit{Lisp Machines} erano dei computer il cui \textit{hardware} era ottimizzato per eseguire programmi scritti in \textit{Lisp}. Negli anni '60 e '70, essendo le risorse ancora limitate, i programmi di intelligenza artificiale erano molto costosi dal punto di vista computazionale. Il linguaggio simbolico \textit{Lisp} era lento perchè i computer commerciali erano progettati per eseguire programmi in \textit{Fortran} e \textit{Assembly}. Perciò vennero costruiti questi computer con lo scopo di rendere la ricerca in questo ambito più efficiente.

        \href{https://en.wikipedia.org/wiki/Lisp_machine}{\underline{Lisp Machine}}

    \item \textbf{Cosa è il modello OCEAN?} \par
        Il modello \textit{OCEAN} è un raggruppamento di cinque tratti che rendono la personlità di una persona unica. Essi sono: apertura all'esperienza, coscienziosità, gradevolezza, estroversione e nevroticismo. Queste caratteristiche individuali possono essere utilizzate per creare dei modelli predittivi in grado di supporre i comportamenti futuri di un soggetto.

        \href{https://en.wikipedia.org/wiki/Big_Five_personality_traits}{\underline{Big Five personality traits}}

    \item \textbf{Il \textit{"No Free Lunch (NFL) Theorem"}} \par
        Questo teorema fu ideato da David Wolpert e William Macready. Esso dice che non esiste un unico migliore algoritmo nè di ottimizzazione nè di \textit{machine learning}.

        \href{https://machinelearningmastery.com/no-free-lunch-theorem-for-machine-learning/}{\underline{No Free Lunch Theorem for Machine Learning}}

    \item \textbf{AlphaGo} \par
        \textit{AlphaGo} è un programma in grado di giocare a \textit{Go}, un gioco inventato in Cina più di 2500 anni fa. È stato realizzato da \textit{Deep Mind} ed è stato in grado di vincere contro i giocatori più forti al mondo. Ha perfino fatto mosse che inizialmente sembravano sconvenienti, ma più avanti nella partita si sono rivelate geniali.
        Il suo successore \textit{AlphaZero}, che ha imparato ha giocare a \textit{Go} in modo completamente autonomo, ha vinto 100 volte su 100 contro \textit{AlphaGo} ed è in grado di giocare anche a scacchi e a \textit{shogi}.

        \href{https://en.wikipedia.org/wiki/AlphaGo}{\underline{AlphaGo} \hspace{1em}}
        \href{https://en.wikipedia.org/wiki/Go_(game)}{\underline{Go (game)}}
\end{itemize}
\end{document}

