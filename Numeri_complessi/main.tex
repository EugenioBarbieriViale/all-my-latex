\documentclass{article}

\usepackage[a4paper, total={6in, 8in}]{geometry}
\usepackage{amsmath}
\usepackage{amsfonts}
\usepackage{amssymb}

\title{Numeri Complessi}
\author{Eugenio Barbieri Viale}
\date{5 novembre 2024}

\begin{document}
\setlength{\parindent}{0pt}
\maketitle

\section*{L'insieme dei numeri complessi}

Con $a \in \mathbb{R}$ e $b \in \mathbb{R}$, un numero complesso $z \in \mathbb{C}$ è scritto come :
$$ (a, b) \Longleftrightarrow a + bi $$

Il suo modulo:
$$ |z| = \sqrt{a^2 + b^2} $$

Il suo coniugato:
$$ z = a + bi \Longleftrightarrow \overline{z} = a - bi $$

\section*{Potenze di $i$}
\begin{align*}
i^0 &= 1 \\
i^1 &= i \\
i^2 &= -1 \\
i^3 &= -i \\
\end{align*}

Questi valori si ripetono con periodo $4$. Quindi:
$$ i^n = i^{n \bmod 4} $$

\section*{Proprietà dei complessi coniugati}
\begin{align*}
    z\overline{z} &= |z|^2 \\
    \frac{1}{z} &= \frac{\overline{z}}{|z|^2} \\
\end{align*}


\section*{Operazioni in $\mathbb{C}$}
Addizione:
$$ (a + bi) + (c + di) = (a + c) + (b + d)i $$

\noindent Sottrazione:
$$ (a + bi) - (c + di) = (a - c) + (b - d)i $$

\noindent Moltiplicazione:
$$ (a + bi) (c + di) = (ac - bd) + (ad + bc)i $$

\noindent Reciproco:
$$\frac{1}{a + bi} = \frac{a - bi}{a^2+b^2} $$

\noindent Divisione:
$$ \frac{a+bi}{c+di} = \frac{(a+bi)(c-di)}{|c+di|^2} $$

\noindent Elevamento al quadrato:
$$ (a+bi)^2 = a^2 - b^2 + 2abi $$

\section*{Rappresentazione trigonometrica}
\begin{align*}
    a =r\cos{\alpha}& \hspace{1em} b =r\sin{\alpha} \\
    r = \sqrt{a^2+b^2}& \hspace{2em} \alpha = \arctan{\frac{b}{a}}
\end{align*}

Quindi un numero complesso può essere scritto come
$$ z = r(\cos{\alpha} + i\sin{\alpha}) $$

\section*{Operazioni tra $z\in\mathbb{C}$ in forma trigonometrica}
Moltiplicazione:
$$ z_1 z_2 = r_1r_2[\cos{(\alpha+\beta)}+i\sin{(\alpha+\beta)}] $$

\noindent Divisione:
$$ \frac{z_1}{z_2} = \frac{r_1}{r_2}[\cos{(\alpha-\beta)}+i\sin{(\alpha-\beta)}] $$

\noindent Reciproco:
$$ \frac{1}{z} = \frac{1}{r}(\cos{\alpha} - i\sin{\alpha}) $$

\noindent Potenze con esponenti $n \in \mathbb{Z}$:
\begin{align*}
    n \geq 0 &\Longrightarrow z^n = r^n(\cos{n\alpha} + i\sin{n\alpha}) \\
    n < 0 &\Longrightarrow \frac{1}{z^n} = \frac{1}{r^n}(\cos{n\alpha} - i\sin{n\alpha})
\end{align*}

\section*{Radici $n$-esime di $z\in\mathbb{C}$}
La radice $n$-esima di un numero complesso ha $n$ valori. \\
Le radici dell'unità immaginaria:
$$ \sqrt[\leftroot{-2}\uproot{2}n]{i} = \cos{\frac{2k\pi}{n}} + i\sin{\frac{2k\pi}{n}} \hspace{1em} \text{con k naturale e} \hspace{1em} k \in [0;n-1]$$
Le radici di $i$, sul piano complesso, si dispongono sulla circonferenza di raggio $1$. Esse formano poligoni regolari inscritti in tale circonferenza.

\vspace{1em}

Le radici di un numero complesso:
$$ \sqrt[\leftroot{-2}\uproot{2}n]{z} = \sqrt[\leftroot{-2}\uproot{2}n]{r}\left[\cos{\frac{\alpha+2k\pi}{n}} + i\sin{\frac{\alpha+2k\pi}{n}}\right] \hspace{1em} \text{con k naturale e} \hspace{1em} k \in [0;n-1]$$
\end{document}

