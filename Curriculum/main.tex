\documentclass[a4paper,9pt]{extarticle}
\usepackage[utf8]{inputenc}
\usepackage{geometry}
\geometry{a4paper, margin=1in}
\usepackage{titlesec}
\usepackage{enumitem}
\usepackage{hyperref}
\usepackage{lipsum}
\usepackage{changepage}

\setlist{noitemsep}
\titleformat{\section}{\large\bfseries}{\thesection}{1em}{}[\titlerule]
\titlespacing*{\section}{0pt}{\baselineskip}{\baselineskip}
\newenvironment{subs}
  {\adjustwidth{2em}{0pt}}
  {\endadjustwidth}


\begin{document}
\pagestyle{empty}

\begin{center}
    \textbf{\Large Eugenio Barbieri Viale}\\[3pt]
    \textbf{Curriculum Vitae}\\[1pt]
    Milan, Italy | \href{mailto:eugenio.barbieri.viale@gmail.com}{eugenio.barbieri.viale@gmail.com} | \href{https://github.com/EugenioBarbieriViale}{Github profile}
\end{center}

\section*{EDUCATION}
    \noindent
    \newline
    \textbf{Elementary School} \\
    Period: 2012/13 - 2016/17 \\
    School name: Istituto Comprensivo Statale Leonardo da Vinci

    \noindent
    \newline
    \textbf{Middle School} \\
    Period: 2017/18 - 2019/20 \\
    School name: Istituto Omnicomprensivo Musicale Statale Giuseppe Verdi

    \noindent
    \newline
    \textbf{High School} \\
    Period: 2020/21 - 2024/25 \\ % fix period 2025/26
    School name: Liceo Scientifico Statale Leonardo da Vinci
    \newline

\section*{ACADEMIC ACHIEVEMENTS}
    \noindent
    \newline
    \textbf{STEM Competitions} \\
    I have participated in math, physics, and computer science competitions offered by my high school, both individually and as a member of a team. \\
    On the 19th of December 2024, I attended the Physics Olympiad and scored fifth in my school.

    \noindent
    \newline
    \textbf{Math \& DL on the Lake} \\
    Between the 12th and 14th of June 2024, I attended a deep learning course organized by the University of Milan that took place at Lake Garda. \\
    I was selected by the organizers among many applicants from different high schools in various cities in northern Italy.

    \noindent
    \newline
    \textbf{Admission to the Polytechnic of Milan} \\
    On the 8th of May 2025, I passed the anticipated admission test. I was in fourth grade, and for this reason I was given direct access to all faculties.
    \newline

\section*{LANGUAGE CERTIFICATES}
    \noindent
    \newline
    \textbf{Deutsches Sprachdiplom (DSD)} \\
    Language: German \\
    Level: DSD I (B1) \\
    Date: March 2024

    \noindent
    \newline
    \textbf{Deutsche Sprachprüfung für den Hochschulzugang (DSH)} \\
    Language: German \\
    Level: DSH-3 (C1.2) \\
    Date: 22nd August 2025

    \noindent
    \newline
    \textbf{Cambridge First Certificate} \\
    Language: English \\
    Level: B2 \\
    Date: [To be completed]

    \noindent
    \newline
    \textbf{Cambridge Proficiency} \\
    Language: English \\
    Level: C2 \\
    Date: 17th and 18th December 2025

    \noindent
    \newline
    \textbf{CUSL - Latin Language Certificate} \\
    Language: Latin \\
    Level: A2 \\
    Date: 17th April 2023
    \newline

\section*{EXTRACURRICULAR ACTIVITIES}
    \noindent
    \newline
    \textbf{Intensive Physics Course} \\
    This course was organized by the Department of Physics of the University of Milan. I participated in my third year between January and February.

    \noindent
    \newline
    \textbf{Tutoring of other Students} \\
    Since my fourth year, I have tutored students from my school and other schools weekly in physics.

    \noindent
    \newline
    \textbf{Volunteer Work} \\
    I have participated in volunteer events such as ``Colletta Alimentare,'' which takes place in November, and ``Un Natale per Tutti,'' organized by the Sant'Egidio community on the 25th of December.
    \newline

\section*{INTERESTS}
    \noindent
    \newline
    \textbf{Programming} \\
    In middle school, I learned how to program on my own as an autodidact. Since then, I have learned different languages such as Python, C, C++, Rust, and Go. \\
    I have created many projects, including physics simulations, cellular automatas, Telegram bots, and machine learning algorithms.

    \noindent
    \newline
    \textbf{Chess and Go} \\
    In my free time, I like to play these two strategy games, because they challenge my problem-solving skills and analytical reasoning. 

\end{document}
