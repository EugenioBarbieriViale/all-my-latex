\documentclass[12pt]{article}

\usepackage[a4paper, total={6in, 8in}]{geometry}
\usepackage{amsmath}

\title{Elettrostatica}
\author{Eugenio Barbieri Viale}
\date{3 maggio 2025}

\begin{document}
\setlength{\parindent}{0pt}
\maketitle

\def \t {\textrightarrow}
\def \v {\vspace{1em}}
\def \bi {\begin{itemize}}
\def \ei {\end{itemize}}
\def \s[#1] {\section*{#1}}
\def \ss[#1] {\subsection*{#1}}
\def \sss[#1] {\subsubsection*{#1}}

Costante dielettrica nel vuoto:
$$ \varepsilon_0 = 8.854 \cdot 10^{-12} C^2/(Nm^2) $$

Negli altri materiali:
$$ \varepsilon = \varepsilon_0 \varepsilon_r $$

\ss[Forza di Coulomb]
$$ |\vec{F_c}| = \frac{1}{4\pi\varepsilon_0}\frac{|q_1||q_2|}{r^2} $$

\ss[Campo elettrico]
Per definizione:
$$ \vec{E} = \frac{\vec{F}}{q_0} $$

Di una carica puntiforme:
$$ |\vec{E}| = \frac{k|q|}{r^2} $$

Generato da un piano infinito uniformemente carico:
$$ |\vec{E}| = \frac{\sigma}{2\varepsilon_0} $$

In un condensatore:
$$ |\vec{E}| = \frac{\sigma}{\varepsilon_0} $$

Generato da un filo conduttore infinitamente lungo e uniformemente carico:
$$ |\vec{E}| = \frac{\lambda}{2\pi\varepsilon_0} \frac{1}{r} $$

Generato da una sfera isolante uniformemente carica:
\begin{equation}
    \begin{cases}
        |\vec{E}| = \frac{Q}{4\pi\varepsilon_0 R^3}r \longrightarrow r \le R\\
        |\vec{E}| = \frac{Q}{4\pi\varepsilon_0} \frac{1}{r^2} \longrightarrow r > R
    \end{cases}
\end{equation}

\ss[Flusso]
Per definizione:
$$ \Phi_s(\vec{E}) = \vec{E} \cdot \vec{S} = |\vec{E}||\vec{S}|\cos{\varphi} $$

Teorema di Gauss: \textit{Il flusso del campo elettrico attraverso una superficie gaussiana è uguale al rapporto tra la carica totale racchiusa nella superficie e la costante dielettrica}
$$ \Phi_s(\vec{E}) = \frac{Q_{racchiusa}}{\varepsilon_0} $$

\ss[Energia potenziale]
La forza di coulomb è conservativa, e l'energia potenziale di un sistema di due cariche puntiformi è:
$$ U = \frac{1}{4\pi\varepsilon_0}\frac{|q_1||q_2|}{r} $$
considerando la distanza infinita come riferimento $U=0$

\v

In un sistema di $n$ cariche, le coppie possibili sono:
$$ N = C_{n,2} = \frac{n!}{2!(n-2)!} = \frac{1}{2}n(n-1) $$

\ss[Potenziale elettrico]
Per definizione:
$$ \Delta V = \frac{\Delta U}{q_0} $$

In un condensatore con le armature a distanza $d$:
$$ U = q|\vec{E}|d \longrightarrow V = |\vec{E}|d $$
considerando l'armatura negativa come $U=0$ e $V=0$

\v

Per cariche puntiformi invece:
$$ V = \frac{1}{4\pi\varepsilon_0}\frac{q}{r} $$
considerando la distanza infinita come riferimento $V=0$

\v

\bi
    \item una carica positiva accelera da una regione con potenziale maggiore a una con potenziale minore (seguendo il campo)
    \item una carica negativa accelera da una regione con potenziale minore a una con potenziale maggiore
\ei

\v

Le superifici equipotenziali sono sempre perpendicolari al campo elettrico \\
Il lavoro che serve per spostare una carica lungo una superficie è nullo $L = 0$, poichè il prodotto scalare tra il vettore campo e il vettore spostamente è nullo $\cos{\varphi} = 0 \rightarrow \varphi = \frac{\pi}{2}$

\v

Il potenziale di una sfera conduttrice ($r$ distanza dal centro, $R$ raggio della sfera):
$$ V = \frac{1}{4\pi\varepsilon_0}\frac{q}{r} \longrightarrow r \ge R $$

All'interno della sfera:
$$ V = \frac{1}{4\pi\varepsilon_0}\frac{q}{R} \longrightarrow r < R $$

Il campo in funzione della variazione di potenziale:
$$ E_s = -\frac{\Delta V}{\Delta s} $$
dove $E_s$ è la componente del campo elettrico sul vettore spostamento $\Delta \vec{s}$

\ss[Circuitazione]
La circuitazione del campo elettrico lungo una curva $\gamma$:
$$ \Gamma_{\gamma}(\vec{E}) = 0 $$

\ss[Capacità]
Per definizione:
$$ C = \frac{q}{V} $$

In una sfera:
$$ C_{sfera} = 4\pi\varepsilon_0 R $$

In un condensatore piano di area $A$ e con le armature a distanza $d$:
$$ C = \frac{\varepsilon_0 \varepsilon_r A}{d} $$
dove $\varepsilon_r$ è la costante dielettrica relativa del materiale tra le armature, \textbf{non} delle armature

\v

L'energia potenziale elettrica immagazzinata in un condensatore con differenza di potenziale $\Delta V$ e carica $q$:
$$ U = \frac{1}{2}q \Delta V $$

In funzione del campo:
$$ U = \frac{1}{2} \varepsilon_0 \varepsilon_r AdE^2 $$

La densità di energia:
$$ \delta_{energia} = \frac{1}{2} \varepsilon_0 \varepsilon_r E^2 $$

\end{document}
