\documentclass[]{article}
\usepackage{titlesec}

\title{Reading Journal}
\author{Eugenio Barbieri Viale}
\date{Summer 2024}

\begin{document}
\maketitle
\titlelabel{\thetitle.\quad}
\tableofcontents
\newpage

\section{Romeo and Juliet}
\subsection*{Summary}
\hspace{1em}
The play "Romeo and Juliet" starts with a street fight, which involves members of two rival families: the house of Montague and the house of Capulet. Shortly after, a ball takes place at the Capulet's residence and Romeo, son of Montague, decides to attend the party. Indeed, he wants to meet a girl of name Rosaline. He then sees Juliet and falls in love with her, not knowing that she is a Capulet. The two exchange sweet words and kiss until Juliet's nurse interrupts them.

Once the party has ended, Romeo goes to the Capulet's garden. Here Juliet stands at her balcony and talks to herself, unaware of the presence of her beloved. She is talking about her love for Romeo and despairs because of the rivalry between their families. Romeo decides to show himself and, after expressing their love for one another, they agree to marry in secret.

Juliet tells the nurse about the marriage and she tries to convince the girl to change her mind. The marriage takes place either way and is consecrated by Friar Laurence.

The problems begin when Tybalt, a Capulet, challenges Romeo to a duel. Romeo refuses and is insulted by his rival. At this point, Romeo's friend Mercutio decides to fight on behalf of him. Mercutio is then wounded to death and Romeo murders Tybalt for revenge. For this reason Romeo gets exiled from Verona and shall never return to the city.

The nurse tells Juliet about these tragic events and she is desperate. Romeo spends the night in Juliet's bedroom and they consummate their marriage. Capulet decides that Juliet is going to marry Paris, whom she meets in church. Juliet asks then Laurence for advice, and they plan to fake her own death.

Romeo, who now is in Mantua, learns about her death. Once he is returned to Verona, he kills himself after seeing Juliet lying as if she were dead. Immediatly after, Juliet wakes up and sees Romeo's body. Racked with pain, she commits suicide with a dagger.

\subsection*{Analysis of Juliet}
\hspace{1em}
Juliet is a sweet and kind girl who has not yet turned fourteen. Despite her age, she is forced to marry a man she hardly knows, a destiny that shares with many other girls of her age. However, she is not naïve nor helpless, but determined and strong. Her actions and rational thinking belong to a grown woman and portray the personality of a modern female.

She does not accept this fate and goes against her father's will. This decision is a symptom of great bravery, which is confirmed in the ending, when she kills herself with a bagger, a much harder and more painful solution than suicide by poison. Moreover, she marries a man whose family is hostile to hers. This choice is not even supported by the closest person to her, the nurse. Nevertheless, she does not change her mind, and this proves her resolution even more.

She is aware of the many difficulties she is going to face and tries to overcome them. Romeo is instead too committed to courting her and to promising his love. In fact, Juliet is the one who actually solves situations. With the help of the friar she engineers a perfect plan and fakes her own death. She accepts this extreme solution, even if it means that she will never see her family again.

This organization will be ruined by Romeo, who is the opposite of Juliet. He is impulsive and irrational, and instead of solving problems, he causes them. He gets exiled from the city and believes in the death of her beloved.

In my opinion, it almost feels like Shakespeare decided to give Juliet's character the personality of a courageous man, while Romeo's character the attributes of a helpless girl. In this play, it seems to me that the role of the hero and the one of the girl to be saved are as if they were swapped.

\subsection*{Questions and Quotes}
\begin{itemize}
    \item \textbf{Who is the antagonist of the story, if there is one?} \par
        In this play I don't think that there is an antagonist. Everyone takes actions that can be justified. The father organizes his daughter marriage following the custom of the time. Tybalt has a duel to settle a question, which was ordinary. So I think that one could say that the villain is the rivalry between the two families, or the impulsiveness of Romeo, or maybe even the "stars", as the introduction of the play may suggest.
    \item \textbf{If Romeo was more rational, would Juliet's plan have worked?} \par
        I believe so, because the mistakes he made could have been avoided only if he was more cold-blooded.
    \item \textbf{Had the couple more alternatives?} \par
        Yes, I think so. For example, they could have run away right after their marriage.

    \item[$-$] \textbf{"Well, you have made a simple choice. You know not how to choose a man."} \par
        I like this two phrases said by the nurse to Juliet, because I find them comical and at the same time deep, representative of the girl's loneliness in her hard decision to marry Romeo.
\end{itemize}


\newpage
\section{The Merchant of Venice}
\subsection*{Summary}
\hspace{1em}
Portia is a young woman who lives in Belmont. She lost her father, whose will is that his daughter will marry the man who guesses in which one of three caskets lies her portrait. Many have tried and failed, and Portia is glad about it, since she found all the attendants repellent, like the Prince of Morocco, who is even insulted by her.

Bassiano, a young Venetian, decides to try. He first needs money and asks his friend Antonio, a merchant. Since he also has no cash, he tells Bassanio to name him as the guarantor and to borrow some money from the Jewish Shylock. Shylock is reluctant but finally accepts, at one condition: if Antonio is not able to pay back, he will take a pound of flesh from him.

Bassiano arrives in Belmont with his friend Gratiano. This time Portia hopes that the suitor succeeds because she is in love with him. After moments of great distress, Bassiano opens the right casket and wins Portia's hand. Gratiano and Nerissa, Portia's servant, fall in love and are going to marry, too.

In the meantime, sad news comes from Venice. Antonio has lost all his fortunes and cannot pay Shylock back. The moneylender is taking him to court and requires his flesh. Bassanio and Gratiano leave for Venice, and Portia gives them money to pay off Antonio's debt. Without them knowing, she also writes to her cousin Bellario, who is a lawyer in Padua.

In the court, Shylock refuses the money. At this point, a laywer from Padua sent by Bellario shows up: nobody knows it is Portia under the guise of a man. She gives a speech in which she appeals to the mercy fo Shylock. However, the moneylender shows none.

During the arrangements for the extraction of the flesh, Portia establishes that no blood shall be shed, since the bond requires exactly one pound of flesh and nothing more. Otherwise Shylock will be sentenced to death.

Shylock is defeated and accepts the money. Bassiano wants to thank the lawyer and after some esitation, he gives him Portia's ring. Once returned to Belmont, Portia reveals herself and makes fun of Bassanio.

\subsection*{Analysis of Portia}
\hspace{1em}
Portia is another example of female character with the traits of a modern woman. Intelligence and wealth are, in my opinion, the two major attributes she has. Her personality derives mainly from these features and they make her character complex and unique.

The first direct consequence is her sense of superiority. She thinks she is better than the others, and she is, but this makes her act arrogantly. For example, when she talks about her suitors at the beginning of the story, it is clear that she despises them. On different occasions she also gives proof of her malice, like during the trial against Shylock, or when she plays a trick on Bassanio at the end and makes fun of him.

Of course she also has many positive attributes. She is generous and gives great amount of money to Bassanio just to help Antonio, even if she does not know him. She uses her cleverness to help and is very good at solving problems. In addition, she is incredibly eloquent and manages to use the correct language of the law during the trial, even if she is completely inexperienced.

Portia and Juliet compared are completely different. Juliet is tender and sweet, while Portia is arrogant and witty. However they have some particulars in common. Both women overcome difficulties, while the men around them are everything but useful. At the beginning, Portia and Juliet also have to marry someone their father chose, but they both rebel against this fate.

I think that Portia could be considered an extraordinary case of modernity. In fact, she does not have to rise up and go against someone's will. She is already more powerful than everyone around her and is in control. This, added to her negative aspects, makes her a multi-dimensional character.

\subsection*{Questions and Quotes}
\begin{itemize}
    % \item \textbf{Did Portia help Bassanio in choosing the right casket?} \par
    \item \textbf{Would Portia still have married Bassanio if he chose the wrong casket?} \par

    \item[$-$] \textbf{"When he is best he is a little worse than a man, and when he is worst he is little better than a beast."} \par
        This phrase is said by Portia to her servant Nerissa. She is talking about the Duke of Saxony’s nephew, to whom she is betrothed by her father. I like this sentence because, in my opinion, it's a perfect example of Shakespeare's ability to play with words.
    \item[$-$] \textbf{"The quality of mercy is not strained"} \par
        This is the beginning of Portia's speech on mercy and is another example of Shakespeare's clever use of words. "Not strained" means that mercy does not have to be forced, but it is also not restrained in many humans.
    \item[$-$] \textbf{"Beshrew your eyes"} \par
        Portia seems to be "mad" at Bassanio's eyes because she fell in love with him, but she does not know if she can marry him. He has not opened the casket yet.
\end{itemize}

\newpage
\section{Much Ado About Nothing}
\subsection*{Summary}
\hspace{1em}
This play is settled in the city of Messina, which is ruled by Leonato. When he receives news that his friend Don Pedro is returning from a successful battle along with Claudio and Benedick, he is with his niece Beatrice.

Beatrice had made the decision not to marry anyone and to remain nubile. On the other hand, Benedick had also stated that he has no interest in marriage. The two argument and make fun of each other all the time. Meanwhile, Claudio and Leonato's daughter Hero decide to get married.

At this point Benedick's friends decide to play a trick on him. They talk about how Beatrice is deeply in love with him just close enough so that he can overhear them. Hero and her friend Ursula do the same with Beatrice. 

The trick works and the two fall in love with each other. At the same time, Don John, who is Don Pedro's evil brother, convinces Claudio that her betrothed has betrayed him. Claudio denounces the inexisting betrayal at the wedding and leaves Hero on the altar.

Hero is now desperate and follows his father's advice to fake her own death. Claudio, who later learns the truth, is in despair and under Leonato's order accepts to marry Hero's cousin.

At the wedding, the new bride is revealed to be Hero, and in the end, Beatrice and Benedick openly declare their feelings for one another. Although they are in love, they continue arguing.

\subsection*{Analysis of Beatrice}
\hspace{1em}
Beatrice is seen by many critics as a protofeminist. In fact, at the beginning of the story, she rejects the idea of getting married. In my opinion, Beatrice's character is modern but in a different way than the previous two.

Unlike Juliet and Portia, she expresses on many occasions the wish of being a man, so that she could take action. She acts and speaks just like a man, and often is impulsive. When Claudio puts shame on Hero, Beatrice gets terribly mad and says to Benedick: \textit{"Kill Claudio."}

I think that Beatrice could be considered almost virile in her attitude and she is even treated like a man. No one forces her to marry someone and Benedick argues with her as if she were a male. The fact of actually argumenting with a woman was, at the time, strange itself. This shows that he respects her and considers her his equal.
% is strange itself, because it means that he considers her his equal

However, she is also witty and sharp in her responses. She is often irritated, because she cannot act as man, and on one occasion she shows the vulnerable side of her nature. She easily falls into the trick and immediately believes in Hero and Ursula's words.

\subsection*{Questions and Quotes}
\vspace{1em}
\begin{itemize}
    \item \textbf{Is it possible to fall in love like this?} \par
        In my opinion, it is very unlikely to fall in love this way. I think that the two were in love from the beginning but they could not admit it, or they were in love subconsciously.
    \item[$-$] \textbf{"I took no more pains for those thanks than you take pains to thank me."} \par
        I like this phrase told by Beatrice because it is almost looks like a tounge twister and the idea of an actor getting this line wrong over and over again makes me laugh. 
    \item[$-$] \textbf{"I had rather hear my dog bark at a crow than a man swear he loves me."} \par
        In found this line comical.
    \item[$-$] \textbf{"I cannot be a man with wishing, therefore I will die a woman with grieving."} \par
        This quote represents, in my opinion, Beatrice's desire of acting like a man at best.
\end{itemize}

\newpage
\section{Hamlet, Prince of Denmark}
\subsection*{Summary}
\hspace{1em}
One night Hamlet, the prince of Denmark, is visited by his father's ghost, who had passed away. The ghost reveals to the prince and his friend Horatio that he had been murdered by his brother Claudius, who now is the king, and wants revenge.

The next day Laertes, the son of the councillor Polonius, departs to study abroad. Before leaving, he speaks with her sister Ophelia and warns her about Hamlet, who is courting her. Polonius is also concerned about Hamlet's love for Ophelia and forces her to reject him.

Hamlet pretends he is gone mad to gain time and think whether he should avenge his father or not. During a meeting with Ophelia, he acts like a fool and she tells her father. Polonius decides to talk to the King.

Ophelia is then forced by her father to return the gifts Hamlet had given her. During the meeting, at which the king and the adviser secretly assist, Hamlet breaks Ophelia's heart by saying that he has never loved her. He also wants her to go to a nunnery.

Hamlet asks then a company of actors to stage a play in which the king gets killed. Claudius runs away and this proves his guilt. The prince's mother Gertrude demands explanations and Polonius, who was overhearing, gets accidentally killed by Hamlet.

After her father's death, Ophelia has become mad and drowns herself in a river. Laertes, who is returned, wants revenge and challenges Hamlet to a duel. They both get mortal wounds, but before dying, Hamlet manages to kill Claudius.

\subsection*{Analysis of Ophelia}
\hspace{1em}
The ideal woman during the Elizabethan age was perceived as naïve and innocent. Ophelia perfectly follows this role, because she is sweet, pure and true.

The most important trait of hers is obedience. Indeed, she always follows the given orders and obeys without protesting, even if this goes against her feelings. Unlike previous female characters, who decide their own husband and rise up against their father's impositions, Ophelia rejects Hamlet's advances because Polonious told her to do so.

I think that Ophelia can be considered the innocent victim of the story, at the mercy of men that care for her but cannot understand her. All try to help her, even Claudius and Gertrude, but in their own self-centered way and moved by pity.

Her madness is highly understandable. She thinks that her beloved is gone mad and she is forced to reject him. Then Hamlet insults her and says that he has never loved her, and at last her father gets killed.

Even when she has become insane, she continues to be pure and harmless, and finally commits the extreme act of suicide.

\subsection*{Questions and Quotes}
\begin{itemize}
    \item[$-$] \textbf{"'Tis in my memory lock'd, And you yourself shall keep the key of it."} \par
        I find this quote very poetic and I really like the metaphor, which compares memory to a casket being locked.
\end{itemize}


\newpage
\section{Macbeth}
\subsection*{Summary}
\hspace{1em}
Macbeth, a brave and valorous warrior at the service of the king of Scotland Duncan, is approached by three terrifying witches. They prophesy that he will become king and that his friend Banquo will be his successor.

The soldier tells his wife, Lady Macbeth, what happened with a letter. She is glad to hear this news and wants him to murder the king, who is going to spend the night in their house. At first he is reluctant, but she convinces him.

The couple plans to kill the king by getting his guards drunk and by leaving the blooded daggers near them while they are asleep. Macbeth is still hesitant, but his wife persuades him by appealing to his manhood and says that she would be able to kill her own child without hesitation.

After committing the murder, the brave warrior is completely shaken, hence Lady Macbeth has to take control of the situation. She takes the daggers from her husband's hands and places them near the guardians. The next day the guards are held responsible and the two are nominated king and queen.

Macbeth's kingdom is unhappy and he turns into a dictator. He kills innocent people and also Banquo, whose ghost is going to haunt him. Due to guilt, Lady Macbeth becomes mad. She hysterically washes her hands, trying to eliminate the smell of blood. Shortly after, she dies.

Macbeth then gets killed and in the end, Duncan's son takes the throne.

\subsection*{Analysis of Lady Macbeth}
\hspace{1em}
Lady Macbeth is one of Shakespeare's most frightening characters and is one of the most famous female villains in english literature. This character changes throughout the play from extremly evil to insane due to the sense of guilt.

At the beginning of the story, she is vicious and can be compared to a witch. She has a terrible influence on her husband, who is turned by her from a valorous warrior to a tyrannical king. In addition, she is the one who wants the murder of Duncan. 

The most visible and predominant feeling she owns is ambition. In order to seize power, she is ready to accomplish very unethical things and even kill. She says she is also ready to murder her own child to demonstrate how weak Macbeth is. She tells him that she would be able to hit her baby in the head while he is smiling at her.

Lady Macbeth is extremly manly in her violence and thirst for success. However, she uses female ways to achieve her purposes. Manipulation is her strongest weapon and she controls her husband to accomplish what she cannot do.

However, the sense of guilt slowly takes over. She becomes mad because of the too many regrets and in the end she presumably commits suicide, since she cannot deal with her crimes. 

Despite sharing the same destiny with Ophelia, the two are the opposite of each other. Ophelia is the victim and goes insane because of the tragedies she has to witness, meanwhile Lady Macbeth is the ruthless committer who causes many deaths and becomes mad because she cannot deal with the sense of guilt.

In conclusion, Lady Macbeth is incredibly evil and nefarious. However, she is not completely inhuman and still has a sense of guilt. Her madness is, in my opinion, the last sparkle of hope in her character. 

\subsection*{Questions and Quotes}
\begin{itemize}
    \item \textbf{Who is more responsible for Duncan's murder: Macbeth or Lady Macbeth?} \par
        In a trial only Macbeth would probably be condemned since he is the actual murderer, but morally speaking I cannot find an answer to this question. Seeing Macbeth as a victim is like believing that German generals during World War II were "just following orders". In fact, he could have easly refused to follow his wife and is the one who phisically does the killing. However, he was also manipulated and the mind behind the murder is Lady Macbeth.
    \item \textbf{Are the prophecies of the witches real?} \par
        The three witches play an important role in the story, but do they really see the future or only shape reality with their words? I think that the second option is more likely. Indeed, their prophecies don't always come true. Banquo is not the successor of the king and Macbeth, instead of keeping the throne, gets killed. Maybe they just tell what their interlocutors want to hear and what they are afraid of under the guise of prophecies.
\end{itemize}


\newpage
\section{Antony and Cleopatra}
\subsection*{Summary}
\hspace{1em}
The play is settled at the time of the Roman Empire and the main characters are Antony, a Roman general, and Cleopatra, queen of Egypt. Antony rules the empire with Octavius and Lepidus, but his love for the queen has made him neglect his duties.

They are together when Antony's wife in Rome rebels against Octavius and dies. Despite Cleopatra's complaints, Antony decides to return to Rome and here marries Octavia for political purposes. When Cleopatra is informed of the marriage by a messenger, she reacts hysterically and attacts him. She then asks about the physical appearance of Octavia.

In the meantime, Antony leaves Rome for Athenes and Octavius takes the lead of the empire by imprisoning Lepidus. Antony is extremely angry and returns to Egypt to prepare an army and fight Octavius back. He follows Cleopatra's strategy but ends up losing. She apologizes and Anotny forviges her quickly.

The situation seems desperate and Octavius promises that he will spare Cleopatra if she betrays her lover. She declines the offer and decides to remain loyal to Antony, but he accuses her anyway.

Antony then loses the war against Octavius and is extremly enraged. He vows to kill Cleopatra and she hides in a monument. She fakes her suicide and Antony kills himself, racked with grief. 

Octavius takes Cleopatra prisoner and in the end she kills herself by being bitten by many poisonous snakes.

\subsection*{Analysis of Cleopatra}
\hspace{1em}
Cleopatra is one of the most famous women in history and is known for her beauty. In this play, she is a well-rounded character and is portrayed as a complex female with a unique personality.

She has all the traits of the ideal queen. She is aristocratic and noble, and her beauty is a powerful political weapon. She is also seductive and manipulative, and she is able to take control over men with her charm. Indeed, Antony forgets about his duties as a governor for her and follows the military strategy she suggests in a battle.

However, she is not as cold-blooded as Lady Macbeth. On many occasions she shows her impulsiveness and lack of self-control. For example, when she learns of the marriage between Antony and Octavia, she overreacts and assaults the guiltless messenger. In addition, she asks a lot of questions concerning Octavia's appearence in an almost childish way. In these moments she has a ridicolous behaviour and displays her emotions in a theatrical way. 

Her personality oscillates between the one of a strong queen and the one of a manipulative seductress. This gives Shakespeare's representation a unique and exclusive touch.

\subsection*{Questions and Quotes}
\begin{itemize}
    \item[$-$] \textbf{"I am sick and sullen."} \par
        It is said by Cleopatra when Antony is trying to inform her of his departure. I like how this phrase sounds.
\end{itemize}

\end{document}
