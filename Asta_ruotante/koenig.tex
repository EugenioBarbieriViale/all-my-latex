\documentclass[]{article}
\usepackage{amsmath}

\title{Asta ruotante}
\author{Eugenio Barbieri Viale}
\date{4 marzo 2024}

\usepackage{listings}
\usepackage{color}

\definecolor{dkgreen}{rgb}{0,0.6,0}
\definecolor{gray}{rgb}{0.5,0.5,0.5}
\definecolor{mauve}{rgb}{0.58,0,0.82}

\lstset{frame=tb,
  language=Python,
  aboveskip=3mm,
  belowskip=3mm,
  showstringspaces=false,
  columns=flexible,
  basicstyle={\small\ttfamily},
  numbers=none,
  numberstyle=\tiny\color{gray},
  keywordstyle=\color{blue},
  commentstyle=\color{dkgreen},
  stringstyle=\color{mauve},
  breaklines=true,
  breakatwhitespace=true,
  tabsize=3
}

\begin{document}
\maketitle

Il centro di massa del sistema assa-proiettile si muove di moto rettilineo uniforme parallelamente alla direzione $y$. Ha quindi legge oraria:
$$y(t)=v_{cm}t$$
La distanza percorsa dal centro di massa in un periodo di rotazione $T$ è quindi:
$$\Delta y=v_{cm}T$$
Fissando l'origine del sistema di riferimento inerziale nel centro geometrico dell'asta prima dell'urto, il centro di massa ha coordinata $x_{cm}$ e velocità $v_{cm}$:
$$x_{cm}=\frac{m}{2(m+M)}l \hspace{2em} v_{cm}=\frac{m}{m+M}v_0$$
Se $\beta$ è il rapporto tra la massa dell'asta $M$ e la massa del proiettile $m$, allora si ottiene che $M=\beta m$. Sostituendo risulta che:
$$x_{cm}=\frac{1}{2(\beta+1)}l \hspace{2em} v_{cm}=\frac{1}{\beta+1}v_0$$
La distanza percorsa dal $CM$ in un periodo di rivoluzione dell'asta è perciò:
$$\Delta y=\frac{1}{\beta+1}v_0 T$$
Ora, sapendo che il sistema assa-proiettile è un sistema isolato, poichè la risultante dei momenti esterni è nulla (le forze tra asta e proiettile sono interne), il momento angolare si conserva. Inoltre, per il teorema di König:
$$L_0=L_f \hspace{2em} \rightarrow \hspace{2em} L_0=L_{trasl}+L_{rot}$$
$L_0$ e $L_{trasl}$ hanno come polo l'orgine del sistema di riferimento, mentre $L_{rot}$ è calcolato rispetto al $CM$. Dato che all'inizio l'asta è ferma:
$$mv_0\frac{l}{2}=(m+M)v_{cm}x_{cm}+I_{cm}\omega$$
Ora è necessario calcolare il momento d'inerzia del sistema rispetto al suo centro di massa. Applicando il teorema di Steiner:
\begin{align}
	I_{cm} &= \frac{1}{12}Ml^2 + Mx_{cm}^2 + m(\frac{1}{2}l-x_{cm})^2 \\
		   &= \frac{\beta}{12}ml^2 + \beta ml^2\frac{1}{4(\beta+1)^2} + \frac{1}{4}ml^2(1-\frac{1}{\beta+1}) \\
		   &= ml^2\left(\frac{\beta}{12}+\frac{\beta}{4(\beta+1)^2}+\frac{1}{4(\beta+1)^2}-\frac{1}{2(\beta+1)}+\frac{1}{4}\right) \\
		   &= ml^2\left(\frac{\beta}{12}+\frac{\beta+1}{4(\beta+1)^2}-\frac{1}{2(\beta+1)}+\frac{1}{4}\right) \\
		   &= ml^2\left(\frac{\beta}{12}+\frac{1}{4(\beta+1)}-\frac{1}{2(\beta+1)}+\frac{1}{4}\right) \\
		   &= ml^2\left(\frac{\beta^2+\beta+3-6+3\beta+3}{12(\beta+1)}\right) \\
		   &= ml^2\frac{\beta^2+4\beta}{12(\beta+1)}
\end{align}
Facendo tutte le sostituzioni necessarie nell'equazione dei momenti angolari:
\begin{align}
	&\frac{1}{2}mlv_0=(1+\beta)m\frac{1}{\beta+1}v_0\frac{1}{2(\beta+1)}l+ml^2\frac{\beta^2+4\beta}{12(\beta+1)}\frac{2\pi}{T} \\
	&\frac{1}{2}v_0=\frac{1}{2(\beta+1)}v_0+\frac{2\pi l\beta(\beta+4)}{12T(\beta+1)} \\
	&\frac{1}{2}(\beta+1)v_0=\frac{1}{2}v_0+\frac{2\pi l\beta(\beta+4)}{12T} \\
	&6(\beta+1)v_0-6v_0=\frac{2\pi l\beta(\beta+4)}{T}
\end{align}
Alla fine si ottiene che il periodo:
$$ T=\frac{\pi l(\beta+4)}{3v_0}$$
Sostituendo nella legge oraria del $CM$ e facendo qualche altra semplificazione, si ottiene:
$$y=\frac{\pi l(\beta+4)}{3(\beta+1)}$$
Ora calcoliamo il rapporto tra la distanza percorsa e la lunghezza $l$ dell'asta, in modo da ottenere una quantità adimensionale:
$$y(\beta)=\frac{\pi \beta+4\pi}{3\beta+3}$$
Questa è evidentemente un'iperbole equilatera riferita agli asintoti. I suoi asintoti sono $x=-1$ e $y=\frac{\pi}{3}$. Ricordando che $\beta>0$, poichè è il rapporto tra due masse, e anche che $y\ge0$, dato che è una distanza, la massima distanza percorsa si trova all'intersezione con l'asse $y$. 
$$\hspace{0.5em}$$
Perciò la massima distanza percorsa durante una rivoluzione dell'asta si trova al tendere a $0$ del rapporto $\beta$ tra le due masse.
Qui il codice della simulazione, che ho realizzato usando la libreria di Python "Pygame".
\lstset{language=Python}
\begin{lstlisting}
import pygame, sys
import numpy as np

pygame.init()
clock = pygame.time.Clock()

X,Y = 1000,1000

screen = pygame.display.set_mode([X,Y])
pygame.display.set_caption("Eugenio Barbieri Viale")

# Grid propeties
x = 0
y = Y
dist_grid = 25

# Ratio of mass of the rod and mass of the bullet
beta = 2

# Half of the length of the rod
R = 230

# Position of the center of mass of the system
x_cm = (R)/(beta+1)

# Initial angles of the to ends of the rod
a1 = 0
a2 = np.pi

# Coordinates of the center of mass with respect to the top left
x0 = X//2 + x_cm
y0 = Y//2

# Distances of the ends from the center of mass
l1 = R + x_cm
l2 = R - x_cm

# The two ends and the cm
cm = pygame.math.Vector2(x0, y0)
p1 = pygame.math.Vector2(0,0)
p2 = pygame.math.Vector2(0,0)

d = R

# Initial position and velocity of the bullet
pos = pygame.math.Vector2(X//2 + d-6, y0 + 200)
vel = 2

v_cm = (vel)/(beta+1) # velocity of the center of mass
va = 3*vel/(R*(beta+4)) # angular velocity

while True:
    for event in pygame.event.get():
        if event.type == pygame.QUIT:
            pygame.quit()
            sys.exit()

    screen.fill((200,200,200))

    cos_norm = abs(np.cos(a1))/np.cos(a1)
    borderX = x0 + np.cos(a1)*(d+10)*cos_norm

    if np.sin(a1) >= 0:
        if pos.y <= y0 and pos.x <= borderX:
            y += v_cm

            a1 += va
            a2 += va

            pos.x = x0 - np.cos(a2)*(d-x_cm)
            pos.y = y0 + np.sin(a2)*(d-x_cm)

        else:
            pos.y -= vel

    if np.sin(a1) < 0:
        if pos.y > y0 and pos.x <= borderX:
            y += v_cm

            a1 += va
            a2 += va

            pos.x = x0 - np.cos(a2)*(d-x_cm)
            pos.y = y0 + np.sin(a2)*(d-x_cm)

        else:
            pos.y -= vel

    p1.x = cm.x - np.cos(a1)*l1
    p1.y = cm.y + np.sin(a1)*l1

    p2.x = cm.x - np.cos(a2)*l2
    p2.y = cm.y + np.sin(a2)*l2

    # Draw grid
    yax = pygame.font.SysFont("Comic Sans MS", 20)
    for i in range(100):
        pygame.draw.line(screen, (0,0,0), (x+dist_grid*i,0), (x+dist_grid*i,Y), 2)
        if y > dist_grid:
            write = yax.render(str(dist_grid*i), 1, (255,255,255))
            screen.blit(write, (X//2+3,y-dist_grid*i-Y//2))
            pygame.draw.line(screen, (0,0,0), (0,y-dist_grid*i), (X, y-dist_grid*i), 2)

    pygame.draw.line(screen, (255,0,0), (X//2,0), (X//2,Y), 2)
    pygame.draw.line(screen, (255,0,0), (0,y-Y//2), (X,y-Y//2), 2)

    # Rod
    pygame.draw.line(screen, (255,255,0), cm, p1, 8)
    pygame.draw.line(screen, (0,0,255), cm, p2, 8)

    # Bullet
    pygame.draw.circle(screen, (255,0,0), pos, 6)

    font = pygame.font.SysFont("Comic Sans MS", 35)
    # Conversion from pixel/frame to pixel/second (60 frames every second)
    write1 = font.render("Angular velocity: " + str(round(va*60,4)) + " rad/s", 1, (255,255,255))
    write2 = font.render("Velocity of the center of mass: " + str(round(v_cm*60,2)) + " px/s", 1, (255,255,255))

    screen.blit(write1, (10,10))
    screen.blit(write2, (10,30))

    pygame.display.flip()
    clock.tick(60)
    pygame.display.update()

pygame.quit()
\end{lstlisting}

\end{document}
