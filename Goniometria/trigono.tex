\documentclass[]{article}

\title{Trigonometria}
\author{Eugenio Barbieri Viale}

\begin{document}
\maketitle
\begin{itemize}
    \item \textbf{Area di un triangolo:}
        $$ A = \frac{1}{2} l_1 l_2 \sin{\alpha} $$
        in cui $\alpha$ è l'angolo compreso tra i due lati.

    \item \textbf{Teorema della corda:}
        $$ \overline{AB} = 2r\sin{\alpha} $$
        in cui $r$ è il raggio e $\alpha$ è un qualunque angolo alla circonferenza che insiste sulla corda.

    \item \textbf{Teorema dei seni:}
        $$ \frac{a}{\sin{\alpha}} = \frac{b}{\sin{\beta}} = \frac{c}{\sin{\gamma}} = 2r $$
        in cui il rapporto è tra ogni lato e il suo angolo opposto. $r$ è il raggio della circonferenza circoscritta al triangolo.

    \item \textbf{Teorema del coseno:}
        $$ a^2 = b^2 + c^2 - 2bc\cos{\alpha} $$
        in cui $\alpha$ è l'angolo compreso tra i lati $b$ e $c$, ovvero l'angolo opposto a $a$.
\end{itemize}
\end{document}
