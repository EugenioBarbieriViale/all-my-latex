\documentclass[]{article}

\title{Gravitazione}
\author{Eugenio Barbieri Viale}
\date{29 maggio 2024}

\begin{document}
\maketitle
Le tre leggi di Keplero:
\begin{itemize}
	\item Prima legge di Keplero: I pianeti si muovono attorno al Sole con orbite ellittiche. Il Sole è uno dei due fuochi dell'ellisse
	\item Seconda legge di Keplero: Il raggio vettore di un pianeta spazza aree uguali in tempi uguali (la velocità aereolare è costante)
	\item Teza legge di Keplero:
		$$\frac{T^2}{a^3}=\frac{4\pi^2}{GM_s}$$
\end{itemize}
Legge di gravitazione universale:
$$F = G\frac{m_1m_2}{r^2} \hspace{1em} con \hspace{1em} G=\mathrm{6.673e}{-11}$$
Confrontando con il peso si ricava la costante:
$$g=G\frac{M_t}{r_t^2}$$
Se un satellite si muove di orbita circolare, la sua velocità è:
$$v = \sqrt{\frac{GM_t}{r}}$$
Nei satelliti geostazionari il loro periodo orbitale $T$ è uguale al periodo di rotazione della Terra intorno al suo asse ($24H$)
Energia cinetica e potenziale gravitazionale:
$$K=\frac{1}{2}mv^2 \hspace{2em} U=-G\frac{m_1m_2}{r}$$
Nel caso orbitale:
$$E = -K \hspace{1em} E = \frac{1}{2}U \hspace{1em} U = -2K$$
Quando un pianeta si muove intorno al Sole si conserva sia l'energia meccanica sia il momento angolare. All'afelio e al perielio:
$$r_{af}v_{af} = r_{per}v_{per}$$
Velocità di fuga:
$$v_f = \sqrt{\frac{2GM_t}{R_t}}$$
Che raggio deve avere un corpo di massa $M$ perchè sia un buco nero (la luce non riesce a fuggire)?
$$\sqrt{\frac{2GM}{R}} = c$$
\end{document}
