\documentclass[12pt]{article}

\usepackage[a4paper, total={6in, 8in}]{geometry}
\usepackage{amsmath}
\usepackage{siunitx}

\title{Termodinamica, Elettrostatica, Elettromagnetismo}
\author{Eugenio Barbieri Viale}
\date{}

\begin{document}
\setlength{\parindent}{0pt}
\maketitle
\tableofcontents

\newpage
\section{Termodinamica}

\subsection{Il primo principio della termodinamica}
$$ \Delta U = Q - L $$
\begin{itemize}
    \item $Q > 0$: il sistema assorbe calore
    \item $Q < 0$: il sistema cede calore
    \item $L > 0$: il sistema compie lavoro 
    \item $Q < 0$: il lavoro è compiuto dall'ambiente sul sistema
\end{itemize}

\subsection{Teoria cinetica per un gas ideale}
$$ n = \frac{N}{N_a} = \frac{m_{totale}}{m_{atomica}} $$
La massa totale è espressa in grammi, mentre quella atomica in dalton ($1u = \num{1.66e-27} kg$)

\vspace{1em}

L'equazione di stato dei gas perfetti:
$$ pV = nRT $$
con $R = 8.314 \hspace{0.5em}J/(molK)$ la costante universale dei gas

\vspace{1em}

Per un gas monoatomico, dove $\overline{K}$ è l'energia cinetica media di ogni molecola, valgono le seguenti relazioni:
$$ pV = \frac{2}{3}N\overline{K} $$
$$ \overline{K} = \frac{3}{2}kT $$ 

\vspace{1em}

Quindi:
$$ v_{qm} = \sqrt{\frac{3kT}{m}} $$
Dove $m$ è la massa di una molecola, $T$ la temperatura e $k = \num{1.38e-23} J/K$ (\textit{costante di Boltzmann}, ovvero $\frac{R}{N_A}$)

\vspace{1em}

In un gas a temperatura $T$, ogni grado di libertà di una molecola è associato a un'energia media pari a:
$$ \frac{1}{2}kT $$

\vspace{1em}

L'energia interna di un gas monoatomico (1) e di un gas biatomico (2):
\begin{align}
    U &= \frac{3}{2}nRT \hspace{1em} \text{3 gradi di libertà}\\
    U &= \frac{5}{2}nRT \hspace{1em} \text{5 gradi di libertà (vibrazione non considerata)}
\end{align}


\subsection{Calore specifico molare}
$$ Q = C_mn\Delta T $$
dove $C_m$ è il calore specifico molare ($J/(mol \cdot K)$)

\vspace{1em}

Gas monoatomico: 
\begin{align*}
    C_p &= \frac{5}{2}R \hspace{1em} \text{pressione costante} \\
    C_v &= \frac{3}{2}R \hspace{1em} \text{volume costante}
\end{align*}

\vspace{1em}

Gas biatomico:
\begin{align*}
    C_p &= \frac{7}{2}R \hspace{1em} \text{pressione costante} \\
    C_v &= \frac{5}{2}R \hspace{1em} \text{volume costante}
\end{align*}

Se si definisce $\gamma$ il loro rapporto:
$$ \gamma = \frac{C_p}{C_v} $$
allora si ottiene che:
\begin{align*}
    \gamma &= \frac{5}{3} \hspace{1em} \text{per un gas perfetto monoatomico} \\
    \gamma &= \frac{7}{5} \hspace{1em} \text{per un gas perfetto biatomico}
\end{align*}

\subsection{Trasformazioni termodinamiche di un gas ideale}
\textbf{Trasformazione isobara}, pressione costante:
$$ L = p\Delta V $$

\textbf{Trasformazione isocora}, volume costante:
$$ L = 0 \longrightarrow \Delta U = Q $$

\textbf{Trasformazione isoterma}, temperatura costante:
$$ L = nRT\ln{\frac{V_f}{V_i}} $$
$$ \Delta U = 0 \longrightarrow Q = L $$

\textbf{Trasformazione adiabatica}, non viene scambiato calore:
\begin{align*}
    L &= C_vn(T_i - T_f) \hspace{1em} \text{se a volume costante} \\
    L &= C_pn(T_i - T_f) \hspace{1em} \text{se a pressione costante}
\end{align*}
$$ Q = 0 \longrightarrow \Delta U = -L $$

\subsection{Relazioni tra grandezze in trasformazioni adiabatiche}
\textbf{Temperatura costante:}
$$ p_i(V_i)^{\gamma} = p_f(V_f)^{\gamma} $$

\textbf{Pressione costante:}
$$ T_i(V_i)^{\gamma-1} = T_f(V_f)^{\gamma-1} $$

\textbf{Volume costante:}
$$ (p_i)^{1-\gamma}(T_i)^{\gamma} = (p_f)^{1-\gamma}(T_f)^{\gamma} $$

\vspace{1em}

con $\gamma$ il rapporto tra $C_p$ e $C_v$, come definito precedentemente.

\subsection{Secondo principio della termodinamica}
\textbf{Enunciato di Kelvin-Planck}: \textit{È impossibile realizzare una macchina termica ciclica il cui unico risultato sia la conversione in lavoro di tutto il calore assorbito da un'unica sorgente} 

\vspace{1em}

\textbf{Enunciato di Clausius}: \textit{È impossibile realizzare una trasformazione il cui unico risultato sia quello di trasferire calore da un corpo più freddo a uno più caldo senza l'apporto di lavoro esterno}

\subsection{Macchine termiche}
$$ Q_c > 0 \hspace{1em} Q_f < 0 \hspace{1em} L > 0 $$
Rendimento di una macchina termica:
$$ \eta = \frac{L}{Q_c} $$
Dato che
$$ L = Q_c + Q_f = Q_c - |Q_f| $$
il rendimento può essere scritto anche:
$$ \eta = 1 - \frac{|Q_f|}{Q_c} $$

\subsection{La macchina di Carnot}
Una trasformazione è reversibile se \textit{può essere invertita riportando il sistema e l'ambiente nello stato iniziale, senza che questo porti a un cambiamento nel sistema o nell'universo.} 

\vspace{1em}

La macchina di Carnot è una macchina termica reversibile, ovvero opera con trasformazioni reversibili. Si dimostra che:
$$ \frac{|Q_f|}{Q_c} = \frac{T_f}{T_c} $$
e ne consegue che:
$$ \eta_{carnot} = 1 - \frac{T_f}{T_c} $$
Questo è il rendimento massimo che una macchina termica può raggiungere operando tra le due temperature.

\vspace{1em}

Il \textit{ciclo di Carnot} è composto da:
\begin{itemize}
    \item espansione isoterma a temperatura costante $T_c$
    \item espansione adiabatica: la temperatura diminuisce da $T_c$ a $T_f$
    \item compressione isoterma a temperatura costante $T_f$
    \item compressione adiabatica: la temperatura ritorna $T_c$
\end{itemize}

\subsection{Macchine frigorifere e pompe di calore}
$$ Q_c < 0 \hspace{1em} Q_f > 0 \hspace{1em} L < 0 $$

Qui vale la relazione:
$$ |Q_c| = |L| + Q_f $$

Il coefficiente di prestazione è:
\begin{align*}
    COP &= \frac{Q_f}{|L|} \hspace{1em} \text{in una macchina frigorifera} \\
    COP &= \frac{|Q_c|}{|L|} \hspace{1em} \text{in una pompa a calore}
\end{align*}

\subsection{Entropia}
\textbf{Riformulazione del secondo principio della termodinamica}: \textit{in un sistema isolato l'entropia è una funzione non decrescente nel tempo}
$$ \frac{dS}{dt} \ge 0 $$

In una trasformazione reversibile:
$$ \Delta S = \frac{Q}{T} $$

L'energia dell'universo non cambia in processi reversibili, mentre aumenta sempre in processi irreversibili. Questi ultimi causano un degrado dell'energia, una cui parte non è più utilizzabile. Il lavoro inutilizzato:
$$ L_{inut} = T_f \Delta S_{univ} $$

L'entropia di un macrostato $m$, dove $\Omega$ è il numero di microstati che corrispondono al macrostato e $k$ è uguale alla costante di Boltzmann:
$$ S_m = k\ln{\Omega_m} $$
da cui, la variazione di entropia dal macrostato $a$ al macrostato $b$:
$$ \Delta S_{a\rightarrow b} = k \ln{\frac{\Omega_b}{\Omega_a}} $$

In un sistema ordinato, in cui un macrostato corrisponde a pochi microstati, si può dimostrare che
$$ \Omega \approx E^N $$
dove $\Omega$ è il numero degli stati disponibili, $E$ è l'energia immagazzinata dal sistema ed $N$ è il numero di molecole.

\newpage
\section{Onde}

\subsection{Cos'è un'onda?}
Un'onda è una perturbazione che si propaga nello spazio e trasporta energia senza che ci sia un trasporto di materia

\subsection{Diversi tipi di onda}
\begin{itemize}
    \item \textbf{onda trasversale}: le particelle del mezzo in cui si propaga l'onda oscillano perpendicolarmente alla direzione di propagazione
    \item \textbf{onda longitudinale}: in un solido elastico, le particelle del mezzo in cui si propaga l'onda oscillano lungo la direzione di propagazione \textit{(come il suono)}
\end{itemize}

\subsection{Caratteristiche}
\begin{align*}
    \lambda &= \text{lunghezza d'onda} = \text{distanza tra due creste} \\
    T &= \text{periodo} = \text{$\Delta t$ in cui viene compiuta un'oscillazione completa} \\
    f &= \text{frequenza} = \frac{1}{T} \\
    v &= \text{velocità di propagazione} = \frac{\lambda}{T} = \lambda f \\
    v &= \sqrt{\frac{F_t}{\mu}} \hspace{1em} \text{con} \hspace{1em} \mu = \frac{m}{L} = \text{densità lineare}
\end{align*}

\subsection{Descrizione matematica di un'onda}
$$ y = A \sin{\left(\omega t \pm kx\right)} $$
$$ \text{dove} \hspace{1em} \omega = \frac{2\pi}{T} \hspace{1em} \text{e} \hspace{1em} k = \frac{2\pi}{\lambda} $$
\begin{itemize}
    \item \textbf{$+$} onda si propaga verso sinistra (\textit{direzione} $-x$)
    \item \textbf{$-$} onda si propaga verso destra (\textit{direzione} $+x$)
\end{itemize}

\subsection{Teorema di Fourier}
\textbf{Enunciato}: \textit{Qualsiasi funzione periodica con frequenza $f$ può essere scritta come somma di funzioni sinusoidali con frequenze che sono multipli di $f$}

\subsection{Il suono}
\subsubsection{L'ampiezza massima}
$$ \Delta p_{max} = 2\pi fdvA $$
in cui $f$ è la frequenza, $d$ è la densità del mezzo, $v$ è la velocità di propagazione dell'onda, $A$ è lo spostamento massimo di una molecola dalla posizione di equilibrio

\subsubsection{Velocità del suono in un gas}
$$ v_{suono} = \sqrt{\gamma k_b \frac{T}{m}} $$
$$ \text{dove} \hspace{1em} \gamma = \frac{c_p}{c_v} $$
\begin{itemize}
    \item $\gamma$ è il rapporto tra il calore specifico molare a pressione costante ($c_p$) e a volume costante ($c_v$)
    \item \textbf{gas monoatomico} \textrightarrow \hspace{1em} $\gamma = \frac{5}{3}$
    \item \textbf{gas biatomico} \textrightarrow \hspace{1em} $\gamma = \frac{7}{5}$
\end{itemize}

\subsubsection{Intensità del suono}
$$ I = \frac{P}{A} = \left[\frac{W}{m^2}\right] $$
dove $P$ è la potenza sonora che attraversa perpendicolarmente una data superficie, $A$ è l'area della superficie

\vspace{1em}

$$ I = \frac{P}{4\pi r^2} $$
se la sorgente emette onde sonore in maniera isotropa vale questa relazione. La superficie $A$ è quella di una sfera e $r$ è il raggio di essa, ovvero la distanza dalla sorgente

\vspace{1em}

$$ I = \frac{\Delta p_{max}^2}{2dv} $$
$$ I = 2\pi^2 f^2 dvA^2 $$

\subsubsection{Livello di intensità sonora}
$$ \beta = 10 \log\frac{I}{I_0} $$
con $I_0$ la soglia minima di intensità sonora udibile

\subsection{Effetto Doppler}
\begin{align*}
    f_r &= f_s \frac{1}{1 - \frac{v_s}{v}} \rightarrow \text{sorgente si avvicina a ricevitore fermo ($f_r$ \textit{aumenta)}} \\
    f_r &= f_s \frac{1}{1 + \frac{v_s}{v}} \rightarrow \text{sorgente si allontana da ricevitore fermo($f_r$ \textit{diminuisce)}} \\
    f_r &= f_s(1 + \frac{v_r}{v}) \rightarrow \text{ricevitore si avvicina a sorgente ferma ($f_r$ \textit{aumenta)}} \\
    f_r &= f_s(1 - \frac{v_r}{v}) \rightarrow \text{ricevitore si allontana da sorgente ferma ($f_r$ \textit{diminuisce)}}
\end{align*}
Caso generale:
$$ f_r = f_s\left(\frac{1 \pm \frac{v_r}{v}}{1 \pm \frac{v_s}{v}}\right) $$

\subsection{Interferenza}
Si ha interferenza costruttiva nei punti di ampiezza massima, cioè quando:
$$ \Delta x = k\lambda \hspace{1em} k \in Z$$

\vspace{1em}

Si ha invece interferenza distruttiva nei punti di ampiezza nulla, cioè quando:
$$ \Delta x = (k+\frac{1}{2})\lambda \hspace{1em} k \in Z $$

$\Delta x = x_1-x_2$ è la differenza di cammino, dove $x_1$ e $x_2$ sono le distanze del punto $P$ dalle sorgenti (luminose o sonore) $S_1$ e $S_2$.

\subsection{La luce come onda}
Se $v$ è la velocità di propagazione, $\lambda$ la lunghezza d'onda e $f$ la frequenza:
$$ v = \lambda f $$
e se il raggio passa da un materiale all'altro, con indici di rifrazione $n$ diversi:
$$ \frac{n_1}{\lambda_1} = \frac{n_2}{\lambda_2} $$

\subsubsection{L'esperimento di Young}
Nei punti di interferenza costruttiva ci sono frange chiare che si trovano ad ampiezze di $\theta$ tali che:
$$ \sin{\theta} = k\frac{\lambda}{d}, \hspace{1em} k \in N$$
con $d$ la distanza tra le due fenditure.

\vspace{1em}

Nei punti di interferenza distruttiva ci sono frange scure che si trovano ad ampiezze di $\theta$ tali che:
$$ \sin{\theta} = (k + \frac{1}{2})\frac{\lambda}{d}, \hspace{1em} k \in N$$

Da questa relazione, ponendo $k = 1$ e approssimando $\sin{\theta} \approx \tan{\theta}$ per $\theta \rightarrow 0 $, si ottiene che:
$$ \lambda = \frac{yd}{L} $$
dove $y$ è la distanza tra la frangia chiara di ordine 0 e di ordine 1, mentre $L$ è la distanza tra lo schermo e la doppia fenditura.

\subsection{Diffrazione}
$$ \sin{\theta} = \frac{\lambda}{D} $$
dove $\theta$ è l'angolo di diffrazione, $D$ è la larghezza della fenditura attraverso la quale il suono o la luce passano.

\vspace{1em}

Per la luce, nel caso di un'unica fenditura, le frange di diffrazione scure si trovano ad ampiezze di $\theta$ tali che:
$$ \sin{\theta} = k\frac{\lambda}{D}, \hspace{1em} k \in N $$

\subsection{Battimenti}
$$ f_{bat} = |f_1 - f_2| $$

\subsection{Onde stazionarie}
\textit{Le onde stazionarie sono onde che non si propagano ma rimangono confinate in una regione}

\subsubsection{Trasversali}
$$ f_n = n \frac{v}{2L} \hspace{1em} n = 1,2,3,\dots $$
Queste frequenze costituiscono la serie armonica. Le onde hanno $n$ ventri

\subsubsection{Longitudinali}
$$ f_n = n \frac{v}{2L} \hspace{1em} n = 1,2,3,\dots $$
il tubo ha lunghezza $L$ e ha le estremità aperte

\vspace{1em}
$$ f_n = (2n-1) \frac{v}{4L} \hspace{1em} n = 1,2,3,\dots $$
il tubo ha lunghezza $L$ e ha un'estremità chiusa

\newpage
\section{Elettrostatica}

\subsection{Costanti}
Costante dielettrica nel vuoto:
$$ \varepsilon_0 = 8.854 \cdot 10^{-12} C^2/(Nm^2) $$

Negli altri materiali:
$$ \varepsilon = \varepsilon_0 \varepsilon_r $$

\subsection{Forza di Coulomb}
$$ |\vec{F_c}| = \frac{1}{4\pi\varepsilon_0}\frac{|q_1||q_2|}{r^2} $$

\subsection{Campo elettrico}
Per definizione:
$$ \vec{E} = \frac{\vec{F}}{q_0} $$

\subsubsection*{Carica puntiforme}
$$ |\vec{E}| = \frac{k|q|}{r^2} $$

\subsubsection*{Piano infinito uniformemente carico}
$$ |\vec{E}| = \frac{\sigma}{2\varepsilon_0} $$

\subsubsection*{Condensatore}
$$ |\vec{E}| = \frac{\sigma}{\varepsilon_0} $$

\subsubsection*{Filo conduttore infinitamente lungo e uniformemente carico}
$$ |\vec{E}| = \frac{\lambda}{2\pi\varepsilon_0} \frac{1}{r} $$

\subsubsection*{Sfera isolante uniformemente carica}
\begin{equation}
    \begin{cases}
        |\vec{E}| = \frac{Q}{4\pi\varepsilon_0 R^3}r \longrightarrow r \le R\\
        |\vec{E}| = \frac{Q}{4\pi\varepsilon_0} \frac{1}{r^2} \longrightarrow r > R
    \end{cases}
\end{equation}

\subsection{Flusso}
Per definizione:
$$ \Phi_s(\vec{E}) = \vec{E} \cdot \vec{S} = |\vec{E}||\vec{S}|\cos{\varphi} $$

\vspace{1em}

Teorema di Gauss: \textit{Il flusso del campo elettrico attraverso una superficie gaussiana è uguale al rapporto tra la carica totale racchiusa nella superficie e la costante dielettrica}
$$ \Phi_s(\vec{E}) = \frac{Q_{racchiusa}}{\varepsilon_0} $$

\subsection{Energia potenziale}
La forza di coulomb è conservativa, e l'energia potenziale di un sistema di due cariche puntiformi è:
$$ U = \frac{1}{4\pi\varepsilon_0}\frac{|q_1||q_2|}{r} $$
considerando la distanza infinita come riferimento $U=0$

\vspace{1em}

In un sistema di $n$ cariche, le coppie possibili sono:
$$ N = C_{n,2} = \frac{n!}{2!(n-2)!} = \frac{1}{2}n(n-1) $$

\subsection{Potenziale elettrico}
Per definizione:
$$ \Delta V = \frac{\Delta U}{q_0} $$

In un condensatore con le armature a distanza $d$:
$$ U = q|\vec{E}|d \longrightarrow V = |\vec{E}|d $$
considerando l'armatura negativa come $U=0$ e $V=0$

\vspace{1em}

Per cariche puntiformi invece:
$$ V = \frac{1}{4\pi\varepsilon_0}\frac{q}{r} $$
considerando la distanza infinita come riferimento $V=0$

\vspace{1em}

\begin{itemize}
    \item una carica positiva accelera da una regione con potenziale maggiore a una con potenziale minore (seguendo il campo)
    \item una carica negativa accelera da una regione con potenziale minore a una con potenziale maggiore
\end{itemize}

\vspace{1em}

Le superifici equipotenziali sono sempre perpendicolari al campo elettrico \\
Il lavoro che serve per spostare una carica lungo una superficie è nullo $L = 0$, poichè il prodotto scalare tra il vettore campo e il vettore spostamente è nullo $\cos{\varphi} = 0 \rightarrow \varphi = \frac{\pi}{2}$

\vspace{1em}

Il potenziale di una sfera conduttrice ($r$ distanza dal centro, $R$ raggio della sfera):
$$ V = \frac{1}{4\pi\varepsilon_0}\frac{q}{r} \longrightarrow r \ge R $$

All'interno della sfera:
$$ V = \frac{1}{4\pi\varepsilon_0}\frac{q}{R} \longrightarrow r < R $$

Il campo in funzione della variazione di potenziale:
$$ E_s = -\frac{dV}{ds} $$
dove $E_s$ è la componente del campo elettrico sul vettore spostamento $d\vec{s}$

\subsection{Circuitazione}
La circuitazione del campo elettrico lungo una curva $\gamma$:
$$ \Gamma_{\gamma}(\vec{E}) = 0 $$

\subsection{Capacità}
Per definizione:
$$ C = \frac{q}{V} $$

In una sfera:
$$ C_{sfera} = 4\pi\varepsilon_0 R $$

In un condensatore piano di area $A$ e con le armature a distanza $d$:
$$ C = \frac{\varepsilon_0 \varepsilon_r A}{d} $$
dove $\varepsilon_r$ è la costante dielettrica relativa del materiale tra le armature, \textbf{non} delle armature

\subsection{Energia immagazzinata in un campo elettrico}
L'energia potenziale elettrica immagazzinata in un condensatore con differenza di potenziale $V$ e carica $q$:
$$ U = \frac{1}{2}qV = \frac{1}{2}CV^2 = \frac{1}{2}\frac{q^2}{C} $$

In funzione del campo:
$$ U = \frac{1}{2} \varepsilon_0 \varepsilon_r AdE^2 $$

La densità di energia:
$$ \delta_{energia} = \frac{1}{2} \varepsilon_0 \varepsilon_r E^2 $$

\newpage
\section{Elettromagnetismo}
\subsection{Autoinduzione}
$$ \Phi(\vec{B}) = LI \hspace{1em} $$
Dove $\Phi(\vec{B})$ è il flusso e $L$ è l'induttanza, in Henry ($H$).

$$ \text{fem} = -L \frac{dI}{dt} $$
Dove la fem è la forza elettromotrice.

$$ L = \mu_0 \mu_r \frac{N^2}{l} A $$
Dove $L$ è l'induttanza di un solenoide e $\mu_r$ è la permeabilità relativa del materiale.

$$ I(t) = \frac{V}{R}(1 - e^{-\frac{t}{L/R}}) $$
Dove $I$ è la corrente in un circuito $RL$ con tensione continua che si genera chiudendo il circuito. Inoltre la costante di tempo si definisce come $\tau = \frac{L}{R}$.

$$ I(t) = \frac{V}{R}e^{-\frac{t}{L/R}} $$
Dove $I$ è la corrente autoindotta che si genera aprendo il circuito. Si chiama extracorrente di apertura.

\subsection{Mutua induzione}
$$ \text{fem}_s = -M \frac{dI_p}{dt} $$
Dove la fem è la forza elettromotrice che la spira primaria (con corrente variabile $I_p$) genera sulla spira secondaria. Anche $M$ si esprime in Henry ($H$).

\subsection{Energia immagazzinata in un campo magnetico}
$$ U = \frac{1}{2}LI^2 $$
Dove $U$ è l'energia immagazzinata in un induttore con induttanza $L$.

$$ \delta_{energia} = \frac{U}{\text{vol}} = \frac{1}{2\mu_0 \mu_r }B^2 $$
\end{document}
